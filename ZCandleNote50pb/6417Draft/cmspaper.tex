\documentclass{cmspaper}

\usepackage{amsmath}
\usepackage{amsfonts}
\usepackage{amssymb}
\usepackage{makeidx}
\usepackage{graphicx}
\usepackage{url}
\usepackage{verbatim}
\usepackage{textcomp}
\usepackage{footnote}
\usepackage{hyperref}

\begin{document}

\begin{titlepage}

   \cmsnote{2010/998}
   \date{\today}

   \title{Study of $Z\rightarrow \mu\mu$ production in association with Jets at $\sqrt{s}=7$ TeV}

   \begin{Authlist}
    A.~Author, B.~Author, C.~Author
       \Instfoot{cern}{CERN, Geneva, Switzerland}
    D.~Author, E.~Author, F.~Author
       \Instfoot{ieph}{Institute of Experimental Physics, Hepcity, Wonderland}
   \end{Authlist}

  \collaboration{CMS collaboration}

  \begin{abstract}
    We describe the study of $Z$+jets production at CMS for
    $\sqrt{s}=7$ TeV collision data.  $Z$ bosons are reconstrcuted
    from pairs of muons. A loose selection is applied to reduce the
    background contamination (mainly due to combinatoric in QCD and
    $t\bar t$ events) and the signal yield is determined using a
    Maximum Likelihood fit. The signal yield is studied as a function
    of the jet multiplicity, describing the ratio of n- to n+1-jet
    multiplicity as $R(n) = \alpha + \beta n$.
  \end{abstract}

\end{titlepage}

\setcounter{page}{2}

\section{Introduction}

% \begin{enumerate}
% \item Clean signature
% \item A good handle on many things - reconstruction efficiency, etc.
% \item B-G scaling.  Expect exponential relation roughly due to the fact that there is one more strong vertex.  Nice to check in 7 TeV running.
% \item Important background in BSM searches, estimate effect due to Znunu
% \item The good old overview of sections
% \end{enumerate}


Despite the crowdy environment of high-energy pp collisions at the LHC,
the study of $Z$ bosons at the LHC is a clean signature to study. 
This is particularly true when the $Z$ is reconstructed in dimuon
final state in CMS, due to the robustness of the muon reconstruction
(for which, unlike the case of muons and taus, the contamination from
fakes is negligible) and the high recontruction efficiency.

Owing to good coverage of the detector, $Z\to \mu\mu$ decays can serve
as candle for detector calibration as well as for efficiency studies
(using tag$\&$ probe techniques). In events where the $Z$ boson is
recoiling against other objects, the energy scale of the recoiling
object can be calibrated by the precise measurement of the momentum of
$Z$.  The study of $Z\to \mu\mu$ production in association with jets
allows to test the Standard Model calculations, obtained with
perturbative QCD (pQCD). The multiplicity of jets is expected to
follow an exponential relation, roughly due to the fact that there is
one more strong vertex in the diagrams.  Previous experiments (TODO:
INSERT CITATION) have verified the exponential relation, and results
from the LHC with 7TeV CM energy can provide more insight in the
underlying physics.

The study of these events can also be used to predict the kinematic
distributions of $Z(\nu\nu)$+jets events, an irreducible background in
many beyond-the-Standard-Model (BSM) searches, since the kinematics is
very similar in the muon channel and in the neutrino channels.  The
knowledge of the branching ratios together with various distributions
from $Z\to\mu\mu$ gives us an estimate of the irreducible background.

We study $Z(\mu\mu)$+jets events applying a loose selection, to reduce
the background to an acceptable level. The events are then classified
according to the jet multiplicity. Using an extended and unbinned
Maximum Likelihood (ML) fit we determine the yields for the
$Z(\mu\mu)$+jets events for different jet multiplicities.  While an
inclusive jet counting is more stable against the modeling of initial
and final state radiation, as well as the description of the
underlying event and the presence of pileup, each inclusive sample
is contained in the lower-multiplicity samples, i.e. the samples are 
correlated. We take the correlation into account by performing
a simultaneous fit of the exclusive samples, writing the yields
as a function of the inclusive yields.

In section 1, section 2, section 3, etc.  (TODO: FILL THIS IN LATER)

\section{Data and Monte Carlo samples}

Proton-proton collision data collected by CMS in 2010 are considered
in this study. The run2010A and run2010B PDMuon samples are combined
to include all the available statistics. The run ranges and datasets
used are summarized in table~\ref{Table_DataSamples}.

\begin{table}[htbp]
   \caption{Data samples (TODO: CALCULATE LUMINOSITY)}
   \centering
   \begin{tabular}{|c|c|c|c|c|c|}
   \hline
   Run range & dataset & integrated luminosity \\\hline
   135821-144114 & /Mu/Run2010A-Sep17ReReco\_v2 &  \\\hline
   146240-149442 & /Mu/Run2010B-PromptReco\_v2 &  \\\hline
   \end{tabular}
   \label{Table_DataSamples}
\end{table}

While the analysis is complitely data driven the design of the
analysis strategy makes use of several signal and background
Monte-Carlo samples, fully-simulated with a GEANT4-based description
of the CMS detector. The signal sample, $Z$ to leptons with jet
production, is generated by MADGRAPH matrix-element generator, while
the parton showered is described through PYTHIA6.  Background
processes such as $W\to \ell \nu$+jets production and $t \bar t$+jets
are also generated by MADGRAPH.  QCD background is generated by
PYTHIA6 generator, requiring at least two muons at generator level.
The list of the used samples is summarized in
table~\ref{Table_MCSamples}.

\begin{table}[htbp]
   \caption{MC samples used.}
   \centering
   \begin{tabular}{|c|c|c|c|c|}
   \hline
   MC & Cross section & Generator & Generator-level cut \\\hline
   $Z \rightarrow l\bar{l}$ + Jets & 2400pb & MADGRAPH & $m_{l\bar{l}} > 50 GeV/c^2$ \\\hline
   $W \rightarrow l\nu_l$ + Jets & 24170pb & MADGRAPH &  \\\hline
   $t\bar{t}$ + Jets & 95pb & MADGRAPH &  \\\hline
   QCD (with two muons) & 48.44mb * 0.0000112 & PYTHIA6 &  \\\hline
   \end{tabular}
   \label{Table_MCSamples}
\end{table}

\section{Selection criteria}

% \begin{enumerate}
% \item Triggers
% \item Muons.
% \item Z selection.
% \item Jet reconstruction.  Different jet algorithms.
% \item Selection tables.
% \item Expected yields for MC samples.
% \end{enumerate}

\subsection{Triggers}

Since the instantaneous luminosity was increasing exponentially during
the course of 2010, the corresponding high-level trigger menu keeps
changing.  In particular, some of the muon triggers with lower $p_T$
threshold got prescaled in later runs.  In each run range, the lowest
unprescaled single-muon HLT path is used.  The triggers used in
different run ranges are summarized in table \ref{Table_TriggerUsed}.
All these triggers are fully efficient with respect to the sample of
events selected by the offline selection, such that the trigger
efficiency is not an issue in the analysis.

\begin{table}[htbp]
   \caption{Lowest unprescaled trigger in different run ranges}
   \centering
   \begin{tabular}{|c|c|}
   \hline
   Run range & Lowest unprescaled trigger \\\hline
   135821-147116 & HLT\_Mu9 \\\hline
   147196-148058 & HLT\_Mu11 \\\hline
   148819-149442 & HLT\_Mu15 \\\hline
   \end{tabular}
   \label{Table_TriggerUsed}
\end{table}

\subsection{Muon selection}

Given the clean signature under study, fully characterize by the
narrow peak in the invariant-mass distribution $M_{ll}$ of the two
muons, the event selection is kept as efficient as possible.  Purity
is recovered in the ML fit, which uses the $M_{ll}$ distribution to
separate signal from background.

The following list of variables are used in the identification of
candidate muons:

\begin{enumerate}
\item Muon type.  The muons to be considered a candidate needs to be
  both a tracker muon (inside out) and a global muon (outside in).
\item Number of valid hits in pixel and strip tracker system.  We
  require that there is at least one hit in any of the pixel detector
  layers, as well as a total of 6 hits in both pixel and silicon strip
  detectors combined.
\item Valid hit in the muon chambers.  At least one valid hit in the
  muon chambers that is consistent with the global muon track fitting
  is required.
\item The reduced $\chi^2$ in the global muon track fitting is
  required to be less than 10.
\item Isolation.  We require that the combined relative isolation is
  smaller than 30\% in a cone size of 0.3.  This is not required to
  obtain a high purity sample, but rather that we want to have a
  data-driven background control sample.  Muons from Z decay a priori
  do not have a preference in terms of isolation with respect to other
  activities in the events, but in background events muons are mostly
  produced in jets and therefore have bad isolations.
\end{enumerate}

In addition to the muon identification criteria, the following
kinematic constraints are required:

\begin{enumerate}
\item For the ``first leg'' (harder) muon, we require $p_T$ greater
  than 15 GeV/c, and pseudorapidity range $|\eta|<2.1$.
\item The $p_T$ requirement of the other muon (softer, ``second leg'')
  is 10 GeV/c, and a slightly larger $\eta<2.4$.
\end{enumerate}

\subsection{Z selection}

In addition to the muon identification and kinematics criteria, the
muon pair is required to be at least 0.01 units apart in $\eta-\phi$
to reject possible doubly-reconstructed muons. Any muon pairs with
invariant mass between 60-120 $GeV/c^2$ are considered a good $Z$
candidate.  When there are more than one candidates in the event, the
candidate with highest transverse momentum is used.

\subsection{Jet reconstruction and selection}

A number of different jet reconstruction algorithms are considered in
this study in order to demonstrate that the scaling is not a
coincidence of certain sub-detectors.

\begin{enumerate}
\item Calorimetric jet (\emph{CaloJet}).  The jet constituents are
  taken from both hadronic calorimeter deposits and electomagnetic
  calorimeter deposits.  The calorimeters are first combined into
  \emph{calotowers} according the the geometrical locations of the
  cells.  Calotowers are then treated as four vector with zero norm to
  be included in the jet clustering algorithm.  The jets need to be
  corrected for energy.  To estimate the effect of the energy
  correction, we also perform the analysis in terms of uncorrected
  energies.  The threshold for counting CaloJets is 30 GeV/c, while
  the threshold for couting uncorrected CaloJets is 20 GeV/c.
\item Uncorrected track jet.  All tracks that pass some basic quality
  selection criteria are served as input to the jet clustering
  algorithm, again treating them as massless.  We require that the
  number of hits along the track is greater than 6, and the normalized
  $\chi^2$ is smaller than 20.  The track vertex, compared to the
  primary vertex, has to be within 0.1 cm in z direction, 0.1 cm in
  magnitude, and 600 $\mu$m in the transverse direction.  Also, if
  there exists another reconstructed vertex that is closer to the best
  primary vertex, the track is not included in the jet clustering.
  The pseudorapidity of the track is required to be $-2.4 < |\eta| <
  2.4$, and the momentum is required to be within range 0.5-500 GeV/c.
  The threshold for track jet counting is 20 GeV/c.
\item Particle flow jet (\emph{PF jet}).  There is an elaborate set of
  algorithms that attempts to reconstruct particle candidates as good
  as possible, and then to start physics analysis from there.  The jet
  algorithms take in the reconstructed candidates as input.  The
  threshold for PF jet counting is 20 GeV/c.  Compatibility with the
  primary vertex is not implemented yet.
\end{enumerate}

\subsection{Selection efficiencies and expected yields}

The selection efficiencies for different MC samples are shown in table
\ref{Table_MCSelectionEfficiencies}.  The expected yields are
summaried in table \ref{Table_MCJetYields} for MC samples for
different jet flavors for 10 $pb^{-1}$.  Note that in the QCD sample
the statistics is low and nothing is left after all selections.

Signal selection efficiencies as a function of different jet
multiplicity events are also calculated and shown in tables
\ref{Table_CaloJetJetEfficiency} and
\ref{Table_UncorrectedCaloJetJetEfficiency}.  The efficiencies across
different jet multiplicity bins are compatible with each other for
different selections, except for a few cases.  The requirement of two
muons which reflects the acceptance due to Z momentum spectrum
produces the difference with respect to different jet multiplicities,
and also that there might be muons produced in jets.  Efficiency from
the isolation requirement is also different since when there are more
jets in the event, the muons from vector boson is more likely
overlapped with any of the jets, hence the smaller selection
efficiency.  The $p_T$ and $\eta$ requirement is similar across all
jet multiplicity bins when there is at least one jet, which is
reflective of the fact that when there are recoiling jets, the
momentum of Z is generally harder than that in events when there is no
recoiling jets.

\begin{table}[htdp]
  \caption{Selection efficiency for signal and background for each
    applied requirement. The values quoted are computed with respect
    to the previously-applied selection.}
 \centering
 \begin{tabular}{|c|c|c|c|c|}
   \hline
   \verb|Sample| & $Z+jets$ & $W+jets$ & $t \bar t+jets$ & $QCD$ \\
   \hline
   \verb|Two Muons|          & $ 57.78 \pm 0.16 $ & $ 4.38 \pm 0.01 $ & $ 30.53 \pm 0.06 $ & $ 57.84 \pm 0.09 $ \\
   \verb|Two Global Muons|   & $ 89.65 \pm 0.29 $ & $ 7.28 \pm 0.04 $ & $ 40.78 \pm 0.12 $ & $ 58.75 \pm 0.13 $ \\
   \verb|Pixel Hits|         & $ 99.45 \pm 0.33 $ & $ 92.99 \pm 0.76 $ & $ 97.46 \pm 0.35 $ & $ 95.63 \pm 0.23 $ \\
   \verb|Tracking Hits|      & $ 99.95 \pm 0.34 $ & $ 97.22 \pm 0.81$ & $ 98.94 \pm 0.36 $ & $ 99.11 \pm 0.24 $ \\
   \verb|Valid Muon Hits|    & $ 97.40 \pm 0.33 $ & $ 67.90 \pm 0.64 $ & $ 86.60 \pm 0.32 $ & $ 78.34 \pm 0.21 $ \\
   \verb|Muon |$\chi^2$      & $ 99.00 \pm 0.34 $ & $ 92.39 \pm 0.96 $ & $ 95.99 \pm 0.38 $ & $ 97.47 \pm 0.27 $ \\
   \verb|Isolation |         & $ 96.94 \pm 0.33 $ & $ 25.90 \pm 0.43$ & $ 14.49 \pm 0.11 $ & $ 28.56 \pm 0.12 $ \\
   $p_T$ \verb|and| $|\eta|$ & $ 91.67 \pm 0.33 $ & $ 0.55 \pm 0.11 $ & $ 32.99 \pm 0.49 $ & $ 0.00 \pm 0.00 $ \\
   \hline
   \verb|Total|              & $ 44.12 \pm 0.14 $ & $ 0.00 \pm 0.00 $ & $ 0.48 \pm 0.01 $ & $ 0.00 \pm 0.00 $ \\
   \hline
   \end{tabular}
\label{Table_MCSelectionEfficiencies}
\end{table}

\begin{table}[htdp]
   \caption{Expected number of events in 2.66 pb$^{-1}$ for signal and background Monte Carlo samples.}
   \centering
       \begin{tabular}{|c|c|c|c|c|}
       \hline
       \verb|Jet Counting| & $Z+jets$ & $W+jets$ & $t \bar t+jets$ & $QCD$ \\
       \hline
       \multicolumn{5}{|c|}{PFJets} \\
       \hline
       $N(\geq 0~jets)$       & $ 938.78 \pm 30.64 $ & $ 0.16 \pm 0.41 $ & $ 1.21 \pm 1.10 $ & $ 0.00 \pm 0.00 $ \\
       $N(\geq 1~jets)$       & $ 140.18 \pm 11.84 $ & $ 0.03 \pm 0.18 $ & $ 1.18 \pm 1.09 $ & $ 0.00 \pm 0.00 $ \\
       $N(\geq 2~jets)$       & $ 25.05 \pm 5.00 $ & $ 0.01 \pm 0.11 $ & $ 0.95 \pm 0.98 $ & $ 0.00 \pm 0.00 $ \\
       $N(\geq 3~jets)$       & $ 4.54 \pm 2.13 $ & $ 0.01 \pm 0.11 $ & $ 0.41 \pm 0.64 $ & $ 0.00 \pm 0.00 $ \\
       $N(\geq 4~jets)$       & $ 0.90 \pm 0.95 $ & $ 0.00 \pm 0.00 $ & $ 0.13 \pm 0.36 $ & $ 0.00 \pm 0.00 $ \\
       \hline
       \multicolumn{5}{|c|}{TrackJets} \\
       \hline
       $N(\geq 0~jets)$       & $ 938.78 \pm 30.64 $ & $ 0.16 \pm 0.41 $ & $ 1.21 \pm 1.10 $ & $ 0.00 \pm 0.00 $ \\
       $N(\geq 1~jets)$       & $ 99.29 \pm 9.96 $ & $ 0.03 \pm 0.18 $ & $ 1.08 \pm 1.04 $ & $ 0.00 \pm 0.00 $ \\
       $N(\geq 2~jets)$       & $ 13.90 \pm 3.73 $ & $ 0.01 \pm 0.11 $ & $ 0.67 \pm 0.82 $ & $ 0.00 \pm 0.00 $ \\
       $N(\geq 3~jets)$       & $ 2.24 \pm 1.50 $ & $ 0.00 \pm 0.00 $ & $ 0.22 \pm 0.47 $ & $ 0.00 \pm 0.00 $ \\
       $N(\geq 4~jets)$       & $ 0.35 \pm 0.59 $ & $ 0.00 \pm 0.00 $ & $ 0.06 \pm 0.24 $ & $ 0.00 \pm 0.00 $ \\
       \hline
       \multicolumn{5}{|c|}{CaloJets} \\
       \hline
       $N(\geq 0~jets)$       & $ 938.78 \pm 30.64 $ & $ 0.16 \pm 0.41 $ & $ 1.21 \pm 1.10 $ & $ 0.00 \pm 0.00 $ \\
       $N(\geq 1~jets)$       & $ 211.96 \pm 14.56 $ & $ 0.07 \pm 0.26 $ & $ 1.19 \pm 1.09 $ & $ 0.00 \pm 0.00 $ \\
       $N(\geq 2~jets)$       & $ 43.27 \pm 6.58 $ & $ 0.03 \pm 0.16 $ & $ 1.02 \pm 1.01 $ & $ 0.00 \pm 0.00 $ \\
       $N(\geq 3~jets)$       & $ 8.63 \pm 2.94 $ & $ 0.01 \pm 0.11 $ & $ 0.52 \pm 0.72 $ & $ 0.00 \pm 0.00 $ \\
       $N(\geq 4~jets)$       & $ 1.81 \pm 1.35 $ & $ 0.00 \pm 0.00 $ & $ 0.21 \pm 0.46 $ & $ 0.00 \pm 0.00 $ \\
       \hline
       \multicolumn{5}{|c|}{UncorrectedCaloJets} \\
       \hline
       $N(\geq 0~jets)$       & $ 938.78 \pm 30.64 $ & $ 0.16 \pm 0.41 $ & $ 1.21 \pm 1.10 $ & $ 0.00 \pm 0.00 $ \\
       $N(\geq 1~jets)$       & $ 114.41 \pm 10.70 $ & $ 0.05 \pm 0.21 $ & $ 1.15 \pm 1.07 $ & $ 0.00 \pm 0.00 $ \\
       $N(\geq 2~jets)$       & $ 17.73 \pm 4.21 $ & $ 0.01 \pm 0.08 $ & $ 0.81 \pm 0.90 $ & $ 0.00 \pm 0.00 $ \\
       $N(\geq 3~jets)$       & $ 2.99 \pm 1.73 $ & $ 0.01 \pm 0.08 $ & $ 0.30 \pm 0.55 $ & $ 0.00 \pm 0.00 $ \\
       $N(\geq 4~jets)$       & $ 0.59 \pm 0.77 $ & $ 0.00 \pm 0.00 $ & $ 0.08 \pm 0.29 $ & $ 0.00 \pm 0.00 $ \\
       \hline
       \end{tabular}
\label{Table_MCJetYields}
\end{table}


\begin{table}[htdp]
  \caption{Selection efficiency for signal in different calo jet bin for each
    applied requirement. The values quoted are computed with respect
    to the previously-applied selection.}
 \centering
 \begin{tabular}{|c|c|c|c|c|c|}
   \hline
   \verb|Jet count| & $\ge 0 jets$ & $\ge 1 jet$ & $\ge 2 jets$ & $\ge 3 jets$ & $\ge 4 jets$ \\
   \hline
   \verb|Two Muons|          & $ 57.78 \pm 0.16 $ & $ 69.94 \pm 0.41 $ & $ 75.71 \pm 0.98 $ & $ 79.17 \pm 2.29 $ & $ 80.14 \pm 4.99 $ \\
   \verb|Two Global Muons|   & $ 89.65 \pm 0.29 $ & $ 89.98 \pm 0.59 $ & $ 89.59 \pm 1.27 $ & $ 89.95 \pm 2.83 $ & $ 88.36 \pm 5.99 $ \\
   \verb|Pixel Hits|         & $ 99.45 \pm 0.33 $ & $ 99.38 \pm 0.67 $ & $ 99.43 \pm 1.45 $ & $ 99.38 \pm 3.21 $ & $ 99.27 \pm 6.95 $ \\
   \verb|Tracking Hits|      & $ 99.95 \pm 0.34 $ & $ 99.95 \pm 0.67$ & $ 99.93 \pm 1.46 $ & $ 99.84 \pm 3.23 $ & $ 100.00 \pm 7.01 $ \\
   \verb|Valid Muon Hits|    & $ 97.40 \pm 0.33 $ & $ 97.05 \pm 0.66 $ & $ 97.06 \pm 1.43 $ & $ 97.17 \pm 3.17 $ & $ 95.82 \pm 6.79 $ \\
   \verb|Muon |$\chi^2$      & $ 99.00 \pm 0.34 $ & $ 98.77 \pm 0.68 $ & $ 98.66 \pm 1.47 $ & $ 98.38 \pm 3.24 $ & $ 98.46 \pm 7.08 $ \\
   \verb|Isolation |         & $ 96.94 \pm 0.33 $ & $ 90.77 \pm 0.64$ & $ 88.05 \pm 1.36 $ & $ 85.16 \pm 2.94 $ & $ 84.38 \pm 6.36 $ \\
   $p_T$ \verb|and| $|\eta|$ & $ 91.67 \pm 0.33 $ & $ 89.06 \pm 0.66 $ & $ 88.56 \pm 1.45 $ & $ 89.90 \pm 3.31 $ & $ 90.43 \pm 7.29 $ \\
   \hline
   \verb|Total|              & $ 44.12 \pm 0.14 $ & $ 48.44 \pm 0.32 $ & $ 50.32 \pm 0.74 $ & $ 51.72 \pm 1.70 $ & $ 50.60 \pm 3.63 $ \\
   \hline
   \end{tabular}
\label{Table_CaloJetJetEfficiency}
\end{table}


\begin{table}[htdp]
  \caption{Selection efficiency for signal in different uncorrected calo jet bin for each
    applied requirement. The values quoted are computed with respect
    to the previously-applied selection.}
 \centering
 \begin{tabular}{|c|c|c|c|c|c|}
   \hline
   \verb|Jet count| & $\ge 0 jets$ & $\ge 1 jet$ & $\ge 2 jets$ & $\ge 3 jets$ & $\ge 4 jets$ \\
   \hline
   \verb|Two Muons|          & $ 57.78 \pm 0.16 $ & $ 71.88 \pm 0.57 $ & $ 77.08 \pm 1.55 $ & $ 80.94 \pm 4.09 $ & $ 81.76 \pm 9.35 $ \\
   \verb|Two Global Muons|   & $ 89.65 \pm 0.29 $ & $ 89.93 \pm 0.78 $ & $ 89.42 \pm 1.97 $ & $ 89.42 \pm 4.89 $ & $ 89.21 \pm 11.02 $ \\
   \verb|Pixel Hits|         & $ 99.45 \pm 0.33 $ & $ 99.38 \pm 0.89 $ & $ 99.36 \pm 2.25 $ & $ 99.37 \pm 5.59 $ & $ 100.00 \pm 12.70 $ \\
   \verb|Tracking Hits|      & $ 99.95 \pm 0.34 $ & $ 99.96 \pm 0.90$ & $ 99.90 \pm 2.27 $ & $ 99.84 \pm 5.63 $ & $ 100.00 \pm 12.70 $ \\
   \verb|Valid Muon Hits|    & $ 97.40 \pm 0.33 $ & $ 96.77 \pm 0.88 $ & $ 97.01 \pm 2.22 $ & $ 97.46 \pm 5.53 $ & $ 96.77 \pm 12.39 $ \\
   \verb|Muon |$\chi^2$      & $ 99.00 \pm 0.34 $ & $ 98.59 \pm 0.90 $ & $ 98.57 \pm 2.28 $ & $ 99.02 \pm 5.67 $ & $ 100.00 \pm 12.91 $ \\
   \verb|Isolation |         & $ 96.94 \pm 0.33 $ & $ 89.80 \pm 0.85$ & $ 87.11 \pm 2.09 $ & $ 86.66 \pm 5.16 $ & $ 86.67 \pm 11.61 $ \\
   $p_T$ \verb|and| $|\eta|$ & $ 91.67 \pm 0.33 $ & $ 87.02 \pm 0.87 $ & $ 88.69 \pm 2.27 $ & $ 92.21 \pm 5.80 $ & $ 91.35 \pm 12.96 $ \\
   \hline
   \verb|Total|              & $ 44.12 \pm 0.14 $ & $ 47.87 \pm 0.43 $ & $ 50.55 \pm 1.16 $ & $ 55.37 \pm 3.13 $ & $ 55.88 \pm 7.16 $ \\
   \hline
   \end{tabular}
\label{Table_UncorrectedCaloJetJetEfficiency}
\end{table}


\section{Fit strategy}

The signal yield is extracted using an extended and unbinned ML
fit. Events are assigned to different dataset, according to their jet
multiplicity. The jet counting is performed exclusively, eccept for
the largest considered multiplicity (for which the inclusive counting
is used). For each jet multiplicity we write the likelihood function 
as:
\begin{equation}
{\cal L}(n_j) = \frac{e^{-N^{n_j}_S-N^{n_j}_B}}{N!}\prod_i{N^{n_j}_S P^S(M^i_{ll}) + N^{n_j}_B P^B(M^i_{ll})}
\end{equation}
where $N$ is the observed number of events, $N_S$ ($N_B$) is the expected
number of signal (background) events, $M_{ll}$ is the invariant mass
of the two muons for the event number $i$, and $P^S(M_{ll})$ ($P^B(M_{ll})$)
is the signal (background) probability density function describing the
$M_{ll}$ disctribution. 

The different samples (for the different jet multiplicities) are
simultaneously considered in the ML fit. The signal and background
observables are written as:
\begin{equation}
N^{n_j} = N^{\geq n_j}- N^{\geq n_j+1}
\end{equation}
for $n_j$ smaller than the largest considered jet multiplicity $n^{max}_j$
and
\begin{equation}
N^{n^{max}_j} = N^{\geq n^{max}_j}
\end{equation}
for the last bin (for which the inclusive jet counting is used).  The
likelihood is the maximized as a function of the inclusive yields
$N^{\geq n_j}$. This strategy allows to measure the signal yield as a
function of the jet multiplity while taking into account the
correlation among the different data samples. The output of this fit is the
model-independent result of this analysis.

In addition, we test the scaling of the signal yield on the number of jets
writing the ratio:
\begin{equation}
R(n) = \frac{N^{n_j}}{N^{n_j+1}} = \alpha + \beta n .
\label{eq:scaling}
\end{equation}
At leading order one would expect $\beta=0$, accoridng to the
Berends-Giele scaling.  On the other hand, data from teh Tevatron
suggest a deviation from this scaling, well described by a linear
dependence on the number of jet (possibly due to a phase-space
reduction at larger jet multiplicity). By enforcing the dependence of
Eq.~\ref{eq:scaling} we can test the linearity of the scaling at 7
TeV. More details are given in sec.~\ref{sec:simfit}.

% \begin{enumerate}
% \item Signal and background shape assumption.
% \item Getting alpha L from data.
% \item Bias from assuming this signal shape.
% \item Simultaneous fit to all jet multiplicities.
% \end{enumerate}

\subsection{Signal and background shape}

The signal PDF is parameterized by a Crujiff function, defined as follows:

\begin{equation}
F_S(M_{ll}; m, \sigma_L, \sigma_R, \alpha_L, \alpha_R) = N_s e^{-\dfrac{(M_{ll} - m)^2}{2 \sigma^2 + \alpha^2 (M_{ll} - m)^2}},\nonumber
\end{equation}

where $\sigma = \sigma_L (\sigma_R)$ for $M_{ll} < m (M_{ll} \geq m)$
and $\alpha = \alpha_L (\alpha_R)$ for $M_{ll} < m (M_{ll} \geq m)$.
The background sample is dominated by QCD events for low jet
multiplicity, while also $t\bar{t}$ events play an important role at
large jet multiplicity.  The background distribution is described by
falling exponential, which correctly describes both the samples.  The
exponent of the background distribution is floated in the fit. The
difference in exponent between QCD and $t \bar t$ is taken account as
a systematic error.

\subsection{Bias from assuming signal shape with wrong parameters}

It is necessary to test whether the signal shape assumption causes any
bias on the extracted signal yield.  Specifically, if we were to fix
any of the parameters in the fit, what is the bias caused by a badly
chosen set of parameters.  A set of toy experiments are carried out by
generating toy samples according to the MC signal shape and refit
using slightly wrong parameters.  Each parameter is varied by 5\%, 2\%
and 1\% in both directions while keeping the other parameters floated.
The pull distribution is shown in figure \ref{Figure_BadShapePulls},
while the induced bias is summarized in table
\ref{Table_BadShapeSummary}.

The one parameter that we should be careful about is $\alpha_L$.

\begin{figure}
   \centering
   \includegraphics[width=110mm]{ToBeDrawn}
   \caption{Pull distributions for fit with bad shape parameters}
   \label{Figure_BadShapePulls}
\end{figure}

\begin{table}[htdp]
 \caption{Bias induced by assuming bad signal shape parameters}
 \centering
 \begin{tabular}{|c|c|c|c|c|c|}
 \hline
 \end{tabular}
 \label{Table_BadShapeSummary}
\end{table}

\subsection{Getting $\alpha_L$ from data}

As shown above, among the four parameters the one with larger effect
if the value assumed is wrong is $\alpha_L$.  Since the background
shape is a falling spectrum and the $\alpha_L$ parameter in the signal
shape controls the size of tail in the low-mass side, it is needed
that we fix at least this parameter during the fit.  Otherwise this
parameter will adjust so that the background events are included in
signal yield also, creating bias.

One way to extract the $\alpha_L$ is to require tight isolation on
muons to get a high purity signal sample, and then fit to get the
shape parameter.  In MC signal sample fits are done varying isolation
level, and the result summarized in figure
\ref{Figure_AlphaLVsIsolation}.  The shape depend only slightly on the
isolation.  Therefore it is fine to extract the shape parameter from
data directly without using MC shape.

\begin{figure}
   \begin{center}
   \includegraphics[width=110mm]{AlphaLVsIsolation_ReasonableIsolation}
   \caption{Shape parameter varying relative combined isolation level on muons.}
   \label{Figure_AlphaLVsIsolation}
   \end{center}
\end{figure}

\subsection{Simultaneous fit to all jet multiplicity bins}
\label{sec:simfit}

A maximum likelihood fit is performed simulatneously for all different
jet counts, with the total likelihood written as

\begin{eqnarray}
L = (Prefactor) \times \displaystyle\sum_i \{ \displaystyle\sum_{n_{jet}=1}^{n_{max} - 1} \left( (N_{S, n_{jet}} - N_{S, n_{jet} + 1}) F_S(M_{ll}^i) + N_{B, n_{jet}} F_{B, n_{jet}}(M_{ll}^i) \right)
\delta_{n_{jet}, n_{jet}^i}\nonumber\\
+ \left( N_{S, n_{max}} F_S(M_{ll}^i) + N_{B, n_{max}} F_{B, n_{max}}(M_{ll}^i) \right) \theta(n_{jet}^i - n_{max}) \},
\nonumber
\end{eqnarray}

where $F_S$ is the signal PDF, constrained to be the same for all jet
bins.  The background PDF, $F_{B, n_{jet}}$, is not constrained to be
the same for different jet bins, and the exponents are left floating
in the fit.  Each term (except the last) is constrained to be with the
same exclusive jet bin through the Kronecker delta function
$\delta_{n_{jet}, n_{jet}^i}$.  The last jet bin is inclusive,
including all number of jets greater or equal to $n_{max}$,
constrained by the step function $\theta(n_{jet}^i - n_{max})$.  All
the parameters for signal PDF are floated except $\alpha_L$.

The assumption that signal shape is fixed across all jet multiplicity
bins can be checked from the MC signal sample.  Shape parameters for
each jet multiplicity bin for different flavors of jets are summarized
in table \ref{Table_MCSignalShape}.


\begin{table}[htdp]
   \caption{MC signal shape dependence across different jet multiplicities}
       \centering
       \begin{tabular}{|c|c|c|c|c|c|}
       \hline
       \verb|Jet Counting| & m & $\alpha_L$ & $\alpha_R$ & $\sigma_L$ & $\sigma_R$ \\
       \hline
       \multicolumn{6}{|c|}{PFJets} \\
       \hline
       $N(\geq 0~jets)$       &  \\
       $N(\geq 1~jets)$       & \\
       $N(\geq 2~jets)$       & \\
       $N(\geq 3~jets)$       & \\
       $N(\geq 4~jets)$       & \\
       \hline
       \multicolumn{6}{|c|}{TrackJets} \\
       \hline
       $N(\geq 0~jets)$       & \\
       $N(\geq 1~jets)$       & \\
       $N(\geq 2~jets)$       & \\
       $N(\geq 3~jets)$       & \\
       $N(\geq 4~jets)$       & \\
       \hline
       \multicolumn{6}{|c|}{CaloJets} \\
       \hline
       $N(\geq 0~jets)$       & \\
       $N(\geq 1~jets)$       & \\
       $N(\geq 2~jets)$       & \\
       $N(\geq 3~jets)$       & \\
       $N(\geq 4~jets)$       & \\
       \hline
       \multicolumn{6}{|c|}{UncorrectedCaloJets} \\
       \hline
       $N(\geq 0~jets)$       & \\
       $N(\geq 1~jets)$       & \\
       $N(\geq 2~jets)$       & \\
       $N(\geq 3~jets)$       & \\
       $N(\geq 4~jets)$       & \\
       \hline
       \end{tabular}
   \label{Table_MCSignalShape}
\end{table}

\section{Data}

% \begin{enumerate}
% \item Fitted shape parameters from tight isolation cut
% \item Result of the simultaneous fit.
% \end{enumerate}

Result of fit to data signal shape with tight isolation requirement is
shown in table \ref{Table_DataAlphaL}.  The ``tight isolation'' here
refers to 5\% in units of relative combined isolation.  Note that only
the value of $\alpha_L$ is used in subsequent fits.  All others are
left floating.

\begin{table}[htbp]
\caption{Shape parameters requiring one jet of different flavors fitting to data with tight isolation cut}
\centering
   \begin{tabular}{|c|c|c|c|c|c|}
      \hline
      & CaloJet & Uncorrected CaloJet & PF Jet & Track jet \\\hline
      $m$ & 90.83 $\pm$ 0.15 & 90.95 $\pm$ 0.18 & 91.04 $\pm$ 0.16 & 90.96 $\pm$ 0.20 \\\hline
      $\alpha_L$ & 0.522 $\pm$ 0.006 & 0.510 $\pm$ 0.008 & 0.488 $\pm$ 0.007 & 0.493 $\pm$ 0.009 \\\hline
      $\alpha_R$ & 0.441 $\pm$ 0.008 & 0.454 $\pm$ 0.009 & 0.448 $\pm$ 0.008 & 0.448 $\pm$ 0.011 \\\hline
      $\sigma_L$ & 2.043 $\pm$ 0.115 & 2.225 $\pm$ 0.145 & 2.342 $\pm$ 0.129 & 2.237 $\pm$ 0.159 \\\hline
      $\sigma_R$ & 2.205 $\pm$ 0.114 & 2.135 $\pm$ 0.138 & 2.071 $\pm$ 0.120 & 2.143 $\pm$ 0.154 \\\hline
   \end{tabular}
   \label{Table_DataAlphaL}
\end{table}

Table \ref{Table_DataExtractedYields} shows the extracted yields from
the simultaneous fit for different jet flavors.  The fit result
together with ratio of N jets to (N+1) jets are shown in figures
\ref{Figure_RatioFromDataCaloJet},
\ref{Figure_RatioFromDataUncorrectedCaloJet},
\ref{Figure_RatioFromDataPFJet} and
\ref{Figure_RatioFromDataTrackJet}.  Distribution of first 4 bins of
jet multiplicity fits are shown in figures \ref{Figure_CaloJetFit},
\ref{Figure_UncorrectedCaloJetFit}, \ref{Figure_PFJetFit} and
\ref{Figure_TrackJetFit}.

\begin{table}
\caption{Data extracted yields}
\centering
   \begin{tabular}{|c|c|c|c|c|}
      \hline
      & CaloJet & Uncorrected CaloJet & PF Jet & Track jet \\\hline
      $N \ge 1$ & 3469 $\pm$ 75 & 2330 $\pm$ 52 & 2874 $\pm$ 63 & 1858 $\pm$ 50 \\\hline
      $N \ge 2$ & 767 $\pm$ 33 & 416 $\pm$ 24 & 622 $\pm$ 28 & 311 $\pm$ 20 \\\hline
      $N \ge 3$ & 159 $\pm$ 15 & 72 $\pm$ 10 & 116 $\pm$ 12 & 41 $\pm$ 8 \\\hline
      $N \ge 4$ & 37 $\pm$ 7 & 11 $\pm$ 5 & 21 $\pm$ 6 & 5 $\pm$ 3 \\\hline
      $N \ge 5$ & 6 $\pm$ 3 & 0 $\pm$ 0 & 2 $\pm$ 2 & 0 $\pm$ 0 \\\hline
   \end{tabular}
   \label{Table_DataExtractedYields}
\end{table}

\begin{figure}[hbtp]
   \begin{center} 
  \includegraphics[width=110mm]{FinalPlot_FloatAll_Calo}
   \caption{Summary of N/(N+1) jet ratio for calo jets}
   \label{Figure_RatioFromDataCaloJet}
   \end{center}
\end{figure}

\begin{figure}[hbtp]
   \begin{center}
   \includegraphics[width=110mm]{FinalPlot_FloatAll_UncorrectedCalo}
   \caption{Summary of N/(N+1) jet ratio for uncorrected calo jets}
   \label{Figure_RatioFromDataUncorrectedCaloJet}
   \end{center}
\end{figure}

\begin{figure}[hbtp]
   \begin{center}
   \includegraphics[width=110mm]{FinalPlot_FloatAll_PF}
   \caption{Summary of N/(N+1) jet ratio for particle-flow jets}
   \label{Figure_RatioFromDataPFJet}
   \end{center}
\end{figure}

\begin{figure}[hbtp]
   \begin{center}
   \includegraphics[width=110mm]{FinalPlot_FloatAll_Track}
   \caption{Summary of N/(N+1) jet ratio for track jets}
   \label{Figure_RatioFromDataTrackJet}
   \end{center}
\end{figure}

\begin{figure}[hbtp]
   \begin{center} 
  \includegraphics[width=110mm]{ToBeDrawn}
   \caption{First 4 jet multiplicity bins in calo jet counting}
   \label{Figure_CaloJetFit}
   \end{center}
\end{figure}

\begin{figure}[hbtp]
   \begin{center} 
  \includegraphics[width=110mm]{ToBeDrawn}
   \caption{First 4 jet multiplicity bins in uncorrected calo jet counting}
   \label{Figure_UncorrectedCaloJetFit}
   \end{center}
\end{figure}

\begin{figure}[hbtp]
   \begin{center}
  \includegraphics[width=110mm]{ToBeDrawn}
   \caption{First 4 jet multiplicity bins in particle-flow jet counting}
   \label{Figure_PFJetFit}
   \end{center}
\end{figure}

\begin{figure}[hbtp]
   \begin{center} 
  \includegraphics[width=110mm]{ToBeDrawn}
   \caption{First 4 jet multiplicity bins in track jet counting}
   \label{Figure_TrackJetFit}
   \end{center}
\end{figure}

\section{Systematic uncertainties}

% \begin{enumerate}
% \item Jet energy scale uncertainty
% \item Anti-muon selection.  Background control sample
% \item Bias on the simultaneous fit strategy.
% \end{enumerate}

\subsection{Jet energy correction uncertainty}

One major systematic uncertainty arises from the jet energy scale
correction uncertainties, which might affect the counting of jet
multiplicity.  In order to estimate the effect of jet energy
correction, we followed the result from study of jet energy correction
uncertainties (TODO: INSERT CITATION?):

\begin{enumerate}
\item 10\% overall scale uncertainty for CaloJets, and 5\% overall
  scale uncertainty for particle-flow jets
\item $2\% \times |\eta|$ pseudorapidity-dependent uncertainty for
  both types of jets
\item The two items above are assumed to add in quadrature.
\end{enumerate}

Therefore we have artificially varied the jet energy by $1 \sigma$ of
each types (overall and eta-dependent) of uncertainties and redo
counting for all events.  After modifying the jet counts, the same
simultaneous fit is performed for each case to extract signal yield.
Difference of signal yield before and after adjusting energy scale is
quoted as the systematics due to jet energy correction.  The result is
shown in figures \ref{Figure_CaloJetJES} and \ref{Figure_PFJetJES} for
CaloJets and PFJets, respectively.

\begin{figure}[hbtp]
\begin{center}
\includegraphics[width=150mm]{CaloJet_JES}
\caption{Jet energy scale uncertainty estimation for CaloJets}
\label{Figure_CaloJetJES}
\end{center}
\end{figure}

\begin{figure}[hbtp]
\begin{center}
\includegraphics[width=150mm]{PFJet_JES}
\caption{Jet energy scale uncertainty estimation for PF jets}
\label{Figure_PFJetJES}
\end{center}
\end{figure}


\subsection{Background control sample}

The background shape hypothesis is checked by inverting the isolation
on the higher energy leg of muon.  By comparing the QCD MC shape
together with the isolation-inverted MC shape, we can see that the
background shape is invariant to isolation requirement of the muon
(TODO: CAN WE REALLY SEE?).  Figure \ref{Figure_AntiMuonSingleLeg}
shows the comparison between two MC shape and the isolation-inverted
shape from data.  The distributions are normalized.  Indeed the
distribution follows a falling distribution.  Given the expected
number of background events, an exponential should be able to describe
the shape.

\begin{figure}[hbtp]
\begin{center}
\includegraphics[width=150mm]{SingleLegAntiMuon}
\caption{Background control sample with muon isolation inverted.  The distributions are normalized to the same area.}
\label{Figure_AntiMuonSingleLeg}
\end{center}
\end{figure}

\subsection{Bias on the simultaneous fit strategy}

Toys again.....  (TODO)

\section{Summary}

We have presented the measurement of N/(N+1) jet ratio from the dimuon
channel of $Z$ boson events.  The result on the ratio is within
expectation.

...what should we say here?  (TODO)


\begin{thebibliography}{9}
   \bibitem{Yay} {\bf CMS AN-2001/100},
      A. Author, B.Author
      {\em "Hmm"}
\end{thebibliography}
 
\pagebreak

\end{document}
